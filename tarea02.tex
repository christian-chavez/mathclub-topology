\documentclass[b5paper,10pt,twoside]{book}
\input{./preamble.tex}

\begin{document}
\def\thetitle{Tarea 02}
\def\fechaentrega{26 de enero de 2025}


% \pagebreak
% \vspace*{-1.5cm}
% \begin{center}
%     \includegraphics[height=0.075\textheight]{images/LogoClubMate.pdf} 
%         % \hspace{1cm}
%     %     \hfill
%     % \includegraphics[height=0.075\textheight]{images/LogoECMC.pdf}
% \end{center}

% \phantomsection\pdfbookmark{\currfilebase}{\currfilebase}

% \begin{center}
%     {\LARGE
%     Topología \\ Curso Vacacional Enero 2025\\
%     % Teaching Assistant
%     \vspace{0.5cm}
%     \textbf{\thetitle{}}}

%     % \emph{I hear, I forget;
%     % I see, I remember;
%     % I do, I understand.}
    
%     Christian Chávez\\
%     \today
% \end{center}

% % \setcounter{problem}{0}


%=================================

\begin{center}
\begin{minipage}{0.45\textwidth}
    \includegraphics[height=0.075\textheight]{images/LogoClubMate.pdf}
\end{minipage}%
\hfill
\begin{minipage}{0.5\textwidth}
    \begin{flushright}
        {\LARGE Topología} \\ Curso Vacacional -- Enero 2025 \\
        Instructor: Christian Chávez \\
    \end{flushright}
\end{minipage}

\vspace{\baselineskip}
{\huge\thetitle} \\
% \vspace{0.5cm}

7 de enero de 2025

\end{center}



\section*{1.\enspace Continuidad }

\begin{problem}
Demuestra que 
\begin{enumerate}[label=(\roman*)]
    \item La composición de funciones continuas es continua. 
    \item Si \(\mathcal{T}\) y \(\mathcal{U}\) son dos topologías sobre un conjunto \(X\), entonces 
        \[1_X \colon (X,\mathcal{T}) \to (X,\mathcal{U})\]
    es continua si y solamente si \(\mathcal{U}\subset \mathcal{T}\).
\end{enumerate}
\end{problem}

\begin{problem}[Equivalencias de la definición de continuidad]
Demuestra que una función \(f\colon X\to Y\)
es continua si y solo si 
\begin{enumerate}[label=(\roman*)]
    \item para todo \(A\subset Y\) ocurre que \(\text{Cl}\,  f^{-1}(A)\subset f^{-1}(\text{Cl} \, A)\)
    \item dada \(\mathcal{B}\)  una base para la topología de \(Y\), el conjunto \(f^{-1}(B)\) es abierto en \(X\) para todo \(B\in \mathcal{B}\).
\end{enumerate}
\end{problem}

\begin{problem}
Considera el mapa 

\[
f : [0, 2] \to [0, 2] \quad:\quad f(x) = 
\begin{cases} 
x & \text{si } x \in [0, 1), \\ 
3 - x & \text{si } x \in [1, 2].
\end{cases}
\]
Demuestra que \(f\) no es continua.
\end{problem}


\section*{2.\enspace Homeomorfismos}


\begin{problem}
Sea \( f: X \to Y \) una biyección continua. Son equivalentes:
\begin{enumerate}[label=(\roman*)]
    \item \( f \) es un homeomorfismo.
    \item \( f \) es abierta.
    \item \( f \) es cerrada.
\end{enumerate}


\end{problem}

\begin{problem}
 Sea \( f : X \to Y \) un homeomorfismo. Entonces, para todo \( A \subset X \),
\begin{enumerate}[label=(\roman*)]
    \item \( f(\operatorname{Cl} A) = \operatorname{Cl}(f(A)) \)
    \item \( f(\operatorname{Int} A) = \operatorname{Int}(f(A)) \)
    \item \( f(\operatorname{Fr} A) = \operatorname{Fr}(f(A)) \)
    \item \( A \) es un entorno de un punto \( x \in X \) si y solo si \( f(A) \) es un entorno del punto \( f(x) \)
\end{enumerate}
\end{problem}

\section*{3.\enspace Convergencia y continuidad en espacios métricos}

Sean \(  (X, d) \) y \( (Y, \rho) \)
espacios métricos.

\begin{problem}
Sea \( f : X \to Y \).
Demuestra que son equivalentes:
\begin{enumerate}[label=(\roman*)]
    \item \( f \) es continua sobre \( X \) (con la definición de espacios métricos)
    \item Para cualquier abierto \( V \subset Y \), el conjunto  \( f^{-1}(V) \) es abierto en \( X \)
\end{enumerate}

\end{problem}

\noindent
\textbf{Definición.}
Una función \( f : (X, d) \to (Y, \rho) \) es \textbf{lipschitz} si existe una constante \( L > 0 \) tal que 
\[
\rho(f(x), f(y)) \leq L \, d(x, y), \quad \forall x, y \in X.
\]

\begin{problem}
Demuestra que si una función es Lipschitz, entonces es continua.
\end{problem}


\end{document}

