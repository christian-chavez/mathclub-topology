\documentclass[b5paper,10pt,twoside]{book}
\input{./preamble.tex}

\begin{document}
\def\thetitle{Tarea 01 -- Parte 2}


% \pagebreak
% \vspace*{-1.5cm}
% \begin{center}
%     \includegraphics[height=0.075\textheight]{images/LogoClubMate.pdf} 
%         % \hspace{1cm}
%     %     \hfill
%     % \includegraphics[height=0.075\textheight]{images/LogoECMC.pdf}
% \end{center}

% \phantomsection\pdfbookmark{\currfilebase}{\currfilebase}

% \begin{center}
%     {\LARGE
%     Topología \\ Curso Vacacional Enero 2025\\
%     % Teaching Assistant
%     \vspace{0.5cm}
%     \textbf{\thetitle{}}}

%     % \emph{I hear, I forget;
%     % I see, I remember;
%     % I do, I understand.}
    
%     Christian Chávez\\
%     \today
% \end{center}

% % \setcounter{problem}{0}


%=================================

\begin{center}
\begin{minipage}{0.45\textwidth}
    \includegraphics[height=0.075\textheight]{images/LogoClubMate.pdf}
\end{minipage}%
\hfill
\begin{minipage}{0.5\textwidth}
    \begin{flushright}
        {\LARGE Topología} \\ Curso Vacacional -- Enero 2025 \\
        Instructor: Christian Chávez \\
    \end{flushright}
\end{minipage}

\vspace{\baselineskip}
{\huge\thetitle} \\
% \vspace{0.5cm}

7 de enero de 2025

\end{center}



% \section*{1.\enspace Espacios topológicos}


% \begin{problem}
% Sea $X$ un conjunto infinito y $p \in X$. Muestra que
% \begin{align*}
% \mathcal{T}_1 &= \{ A \subseteq X \mid A = \varnothing \, \text{ o } \, X \setminus A \text{ es finito} \}\quad\text{y}\\
% \mathcal{T}_2 &= \{ A \subseteq X \mid A = \varnothing \, \text{ o } \, p \in A \}
% \end{align*}
% son topologías sobre $X$.

% \end{problem}



% \begin{problem}\hfill\null
% \begin{enumerate}[label=(\roman*)]
% \item Sea $X = \{a, b, c, d\}$ dotado de la topología
% \[
% \mathcal{T} = \{\varnothing, X, \{a\}, \{b\}, \{a, c\}, \{a, b\}, \{a, b, c\}\}.
% \]
% Encuentra una base y una subbase para $\mathcal{T}$.

% \item ¿Por qué una topología es una base para ella misma?

% \item Muestra que \(\mathcal{B}  = \{(a, b) \subseteq \mathbb{R} \mid a, b \in \mathbb{Q}\}\)  
% es una  base en $\mathbb{R}$ y que además $\mathcal{T}_{\mathcal{B}} = \mathcal{T}_{\text{usual}}$. (Ayuda: usa el hecho de que todo real es el límite de una secuencia de racionales.)
% \end{enumerate}


% \end{problem}
 

% % \begin{problem}
% %     % Muestra que la topología generada por una base es única.
% % \end{problem}



% \section*{2.\enspace Espacios métricos}

% \begin{problem}
% \begin{enumerate}[label=(\roman*)]
% \item Muestra que la suma/el máximo de métricas es una métrica. En otras palabras, si \(d\) y \(\rho\) son métricas sobre un conjunto \(M\), entonces \(d+\rho\) definida por \((d+\rho)(x,y) = d(x,y) + \rho(x,y)\) es una métrica sobre \(M\). Similarmente, \(\xi(x,y) = \max\left\{ d(x,y) , \rho(x,y) \right\}\) es una métrica sobre \(M\). 

% \item Muestra que las siguientes funciones son métricas en $\mathbb{R}^n$.
% \begin{enumerate}[label=(\alph*)]
%     \item $d(x, y) = \max\{|x_i - y_i| : 1 \leq i \leq n\}$.
%     \item $\rho(x, y) = |x_1 - y_1| + \cdots + |x_n - y_n|$.
% \end{enumerate}
% \end{enumerate}
% \end{problem}


% \begin{problem}
% Sea $d$ una métrica en un conjunto $X$.

% \begin{enumerate}[label=(\roman*)]
% \item Muestra que la función ${D} : X \times X \to [0, +\infty)$ definida por 
% \[
% (x, y) \mapsto \frac{d(x, y)}{1 + d(x, y)}
% \]
% es una métrica en $X$.

% \item Muestra que $d$ y ${D}$ son equivalentes.
 
% \item Muestra que 
% \[
% \rho : X \times X \to [0, +\infty) \; : \;  (x, y) \mapsto \min\{1, d(x, y)\}
% \]
% es una métrica en $X$.

% \item Supón que $f : [0, +\infty) \to [0, +\infty)$ es una función monótona creciente tal que $f(0) = 0$ y 
% \[
% f(x + y) \leq f(x) + f(y)
% \]
% para todo \(x,y\geq 0\).
% Muestra que $f \circ d$ es una métrica en $X$.
% \end{enumerate}
    
% \end{problem}

\section*{3.\enspace Posición de un punto respecto  a un conjunto}


\begin{problem}
    Sea \(A\)  un subconjunto  de un espacio topológico \(X\).
    Verifica que \(A^\circ \subset A\) y \(A\subset \overline{A}\).
    Muestra que \((A^\circ)^\circ = A^\circ\) y \(\overline{\overline{A}} = \overline{A}\). 

\end{problem}


\begin{problem}
    Sean \(A\) y \(B\) un subconjuntos de un espacio topológico.
    Demuestra que 
    \begin{enumerate}[label=(\roman*)]
        \item si \(A\subset B\), entonces \(A^\circ \subset B^\circ\),
        \item si \(A\subset B\), entonces \(\text{Ext}\, {B} \subset \text{Ext}\, {A}\),
        \item si \(A\subset B\), entonces \(\overline{A} \subset \overline{B}\),
        \item \((A\cap B)^\circ  = A^\circ \cap B^\circ \) y \(\overline{A\cup B}=\overline{A}\cup \overline{B}\).
    \end{enumerate}
\end{problem}
    

\begin{problem}
Sea \(A\) un subconjunto de un espacio topológico.
Demuestra que 
\begin{enumerate}[label=(\roman*)]
    \item \(A\) es abierto si y solo si \(A = A^\circ\),
    \item \(A\) es cerrado si y solo si \(A = \overline{A}\),
    \item \(A\) es cerrado si y solo si \(\partial A \subset  {A}\),
    \item \(A\) es cerrado si y solo si \( A' \subset  {A}\).
\end{enumerate}
\end{problem}


\begin{problem} 
    Sean \(A\) y \(B\) un subconjuntos de un espacio topológico.
\begin{enumerate}[label=(\roman*)]
    \item Demuestra que \(\partial A = \partial (X\setminus A)\).
    \item Demuestra que \(\partial A = \overline{A}\cap \overline{X\setminus A}   \).
    \item Encuentra contraejemplos para \((A\cup  B)^\circ  = A^\circ \cup  B^\circ \) y \(\overline{A\cap B}=\overline{A}\cap \overline{B}\).
    \item Prueba que si \(A\) es un subespacio de \(X\) y \(B\subset A\), entonces \(\text{Cl}_A \, B = (\text{Cl}_X B)\cap A\).
    Aquí, \(\text{Cl}_A\) denota la clausura respecto a \(A\), tomado como espacio topológico.
\end{enumerate}
\end{problem}




\end{document}

